\documentclass[12pt,a4paper]{article}

%-------------------------------------
% Paquetes recomendados
%-------------------------------------
\usepackage[spanish]{babel}
\usepackage[utf8]{inputenc}
\usepackage[T1]{fontenc}
\usepackage{amsmath, amssymb, amsfonts}
\usepackage{graphicx}
\usepackage{geometry}
\usepackage{hyperref}
\usepackage{enumitem}
\geometry{margin=2.5cm}

%-------------------------------------
% Datos del documento
%-------------------------------------
\title{\textbf{Tarea: Análisis y Aplicación de Algoritmos de Optimización}}
\author{Carrera: Ciencia de la Computación \\ Asignatura: Modelos de Optimización}
\date{}

%-------------------------------------
% Inicio del documento
%-------------------------------------
\begin{document}
	
	\maketitle
	
	\section*{Objetivo general}
	A partir del siguiente problema de optimización, el estudiante deberá realizar un análisis teórico y experimental que permita:
	\begin{itemize}
		\item Identificar el tipo de problema y sus propiedades analíticas.
		\item Aplicar algoritmos de optimización adecuados (vistos en clase o variaciones justificadas).
		\item Evaluar la calidad de las soluciones obtenidas y el comportamiento de los métodos empleados.
	\end{itemize}
	
	El uso de implementaciones de algoritmos en librerías externas está permitido, siempre que se declare explícitamente la fuente y se cite la documentación correspondiente.
	
	%-------------------------------------
	\section*{Función a optimizar}
	Se considera la siguiente función $f: \mathbb{R}^2 \to \mathbb{R}$:
	
	\begin{equation}
		f(x, y) = -200\, e^{-0.02 \sqrt{x^2 + y^2 + 1}}
	\end{equation}
	
	El objetivo consiste en analizar el comportamiento de esta función y aplicar distintos métodos de optimización para encontrar sus puntos críticos y estacionarios y determinar si corresponden a máximos o mínimos locales o globles.
	
	%-------------------------------------
	\section*{Estructura del informe}
	
	El informe deberá contener las secciones siguientes:
	
	\begin{enumerate}[label=\textbf{\arabic*.}]
		\item \textbf{Datos generales}
		\begin{itemize}
			\item Nombre completo del estudiante
			\item Grupo
		\end{itemize}
		
		\item \textbf{Descripción del problema}
		\begin{itemize}
			\item Planteamiento formal del problema de optimización (minimización o maximización).
			\item Identificación del tipo de modelo: lineal, no lineal, continuo, diferenciable, etc.
		\end{itemize}
		
		\item \textbf{Análisis teórico del modelo}
		\begin{itemize}
			\item Existencia de solución y condiciones de optimalidad.
			\item Propiedades del modelo: convexidad, continuidad, etc.
			\item Interpretación geométrica y analítica de la función.
		\end{itemize}
		
		\item \textbf{Algoritmos utilizados}
		\begin{itemize}
			\item Descripción teórica de al menos dos algoritmos aplicados.
			\item Justificación de la elección de los métodos y su adecuación al problema.
			\item Referencias a librerías o implementaciones externas, incluyendo documentación oficial.
		\end{itemize}
		
		\item \textbf{Análisis comparativo de resultados}
		\begin{itemize}
			\item Comparación del rendimiento de los algoritmos en función de:
			\begin{itemize}
				\item Número de iteraciones o pasos.
				\item Tamaño de paso (\textit{step size}) y parámetros de convergencia.
				\item Diferentes condiciones iniciales.
			\end{itemize}
			\item Discusión sobre eficiencia, precisión y estabilidad.
		\end{itemize}
		
		\item \textbf{Visualización y experimentación}
		\begin{itemize}
			\item Graficación del dominio de la función objetivo \( f(x, y) \).
			\item Representación visual del comportamiento de los algoritmos (trayectorias, contornos, superficie, etc.).
			\item Análisis de zonas con posibles óptimos locales o comportamientos numéricamente inestables.
		\end{itemize}
	\end{enumerate}
	
	%-------------------------------------
	\section*{Condiciones de experimentación}
	\begin{itemize}
		\item Los algoritmos deberán probarse para valores de \( x, y \in [-100, 100] \).
		\item Se recomienda realizar múltiples ejecuciones con diferentes parámetros y puntos iniciales.
		\item Se deben explorar zonas donde los métodos puedan presentar dificultades de convergencia o comportamiento errático.
		\item Documentar y explicar claramente las observaciones obtenidas.
	\end{itemize}
	
	%-------------------------------------
	\newpage
	\section*{Entrega}
	\subsubsection*{El informe deberá incluir:}
	\begin{itemize}
		\item Código fuente implementado.
		\item Resultados numéricos y gráficos representativos.
		\item Conclusiones sobre la efectividad de los métodos aplicados.
	\end{itemize}
	
	\subsubsection*{La evaluación considerará:}
	\begin{itemize}
		\item Claridad y profundidad del análisis teórico.
		\item Justificación de los algoritmos seleccionados.
		\item Calidad y variedad de los experimentos realizados.
		\item Presentación formal y redacción del informe.
	\end{itemize}
	
\end{document}
a